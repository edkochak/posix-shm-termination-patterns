\documentclass[a4paper,12pt]{article}

\usepackage[T2A]{fontenc}
\usepackage[utf8]{inputenc}
\usepackage[russian]{babel}
\usepackage{amsmath}
\usepackage{graphicx}
\usepackage{listings}
\usepackage{color}
\usepackage{hyperref}
\usepackage{geometry}
\usepackage{textcomp} % Для правильного отображения специальных символов

\geometry{left=2cm, right=2cm, top=2cm, bottom=2cm}

\definecolor{codebackground}{rgb}{0.95,0.95,0.95}
\definecolor{codekeywords}{rgb}{0.13,0.29,0.53}
\definecolor{codecomments}{rgb}{0.25,0.50,0.37}
\definecolor{codestrings}{rgb}{0.7,0.1,0.1}

% Настройка корректной поддержки кириллицы в листингах
\lstset{
  backgroundcolor=\color{codebackground},
  keywordstyle=\color{codekeywords}\bfseries,
  commentstyle=\color{codecomments}\itshape,
  stringstyle=\color{codestrings},
  basicstyle=\ttfamily\small,
  breakatwhitespace=false,
  breaklines=true,
  captionpos=b,
  keepspaces=true,
  numberstyle=\tiny\color{codecomments},
  numbers=left,
  numbersep=5pt,
  showspaces=false,
  showstringspaces=false,
  showtabs=false,
  tabsize=2,
  literate=
    {а}{{\cyra}}1
    {б}{{\cyrb}}1
    {в}{{\cyrv}}1
    {г}{{\cyrg}}1
    {д}{{\cyrd}}1
    {е}{{\cyre}}1
    {ё}{{\cyryo}}1
    {ж}{{\cyrzh}}1
    {з}{{\cyrz}}1
    {и}{{\cyri}}1
    {й}{{\cyrishrt}}1
    {к}{{\cyrk}}1
    {л}{{\cyrl}}1
    {м}{{\cyrm}}1
    {н}{{\cyrn}}1
    {о}{{\cyro}}1
    {п}{{\cyrp}}1
    {р}{{\cyrr}}1
    {с}{{\cyrs}}1
    {т}{{\cyrt}}1
    {у}{{\cyru}}1
    {ф}{{\cyrf}}1
    {х}{{\cyrh}}1
    {ц}{{\cyrc}}1
    {ч}{{\cyrch}}1
    {ш}{{\cyrsh}}1
    {щ}{{\cyrshch}}1
    {ъ}{{\cyrhrdsn}}1
    {ы}{{\cyrery}}1
    {ь}{{\cyrsftsn}}1
    {э}{{\cyrerev}}1
    {ю}{{\cyryu}}1
    {я}{{\cyrya}}1
    {А}{{\CYRA}}1
    {Б}{{\CYRB}}1
    {В}{{\CYRV}}1
    {Г}{{\CYRG}}1
    {Д}{{\CYRD}}1
    {Е}{{\CYRE}}1
    {Ё}{{\CYRYO}}1
    {Ж}{{\CYRZH}}1
    {З}{{\CYRZ}}1
    {И}{{\CYRI}}1
    {Й}{{\CYRISHRT}}1
    {К}{{\CYRK}}1
    {Л}{{\CYRL}}1
    {М}{{\CYRM}}1
    {Н}{{\CYRN}}1
    {О}{{\CYRO}}1
    {П}{{\CYRP}}1
    {Р}{{\CYRR}}1
    {С}{{\CYRS}}1
    {Т}{{\CYRT}}1
    {У}{{\CYRU}}1
    {Ф}{{\CYRF}}1
    {Х}{{\CYRH}}1
    {Ц}{{\CYRC}}1
    {Ч}{{\CYRCH}}1
    {Ш}{{\CYRSH}}1
    {Щ}{{\CYRSHCH}}1
    {Ъ}{{\CYRHRDSN}}1
    {Ы}{{\CYRERY}}1
    {Ь}{{\CYRSFTSN}}1
    {Э}{{\CYREREV}}1
    {Ю}{{\CYRYU}}1
    {Я}{{\CYRYA}}1
}

\title{Отчёт по лабораторной работе\\
\large{Взаимодействие клиента и сервера через разделяемую память POSIX}}
\author{Студент}
\date{\today}

\begin{document}

\maketitle
\tableofcontents
\newpage

\section{Введение}

\subsection{Цель работы}
Целью данной лабораторной работы является изучение механизма разделяемой памяти POSIX для организации взаимодействия между процессами, а также реализация различных способов корректного завершения работы процессов.

\subsection{Постановка задачи}
Разработать программы клиента и сервера, взаимодействующих через разделяемую память с использованием функций POSIX. Клиент генерирует случайные числа, а сервер осуществляет их вывод. Необходимо обеспечить корректное завершение работы для одного клиента и одного сервера, при котором удаляется сегмент разделяемой памяти. Предложить и реализовать несколько вариантов корректного завершения.

\section{Теоретическая часть}

\subsection{Разделяемая память POSIX}
Разделяемая память POSIX представляет собой механизм межпроцессного взаимодействия (IPC), который позволяет различным процессам иметь доступ к общему участку памяти. В отличие от других средств IPC (таких как сокеты или каналы), разделяемая память обеспечивает наиболее быстрый обмен данными, так как данные не копируются между процессами.

Основные функции для работы с разделяемой памятью POSIX:
\begin{itemize}
    \item \texttt{shm\_open()} — создание/открытие объекта разделяемой памяти
    \item \texttt{ftruncate()} — установка размера объекта памяти
    \item \texttt{mmap()} — отображение объекта в адресное пространство процесса
    \item \texttt{munmap()} — удаление отображения
    \item \texttt{shm\_unlink()} — удаление объекта разделяемой памяти
\end{itemize}

\subsection{Семафоры POSIX}
Для синхронизации доступа к разделяемой памяти используются семафоры POSIX. Семафор представляет собой счетчик, с помощью которого можно контролировать доступ к общим ресурсам.

Основные функции для работы с семафорами POSIX:
\begin{itemize}
    \item \texttt{sem\_open()} — создание/открытие семафора
    \item \texttt{sem\_wait()} — захват семафора (блокирующая операция)
    \item \texttt{sem\_post()} — освобождение семафора
    \item \texttt{sem\_close()} — закрытие семафора
    \item \texttt{sem\_unlink()} — удаление семафора
    \item \texttt{sem\_timedwait()} — ожидание семафора с таймаутом
    \item \texttt{sem\_getvalue()} — получение значения семафора
\end{itemize}

\section{Практическая часть}

\subsection{Общая структура программы}

Программное решение состоит из следующих компонентов:
\begin{itemize}
    \item Заголовочный файл \texttt{shared\_memory.h} с общими определениями
    \item Реализация клиента для трех вариантов завершения
    \item Реализация сервера для трех вариантов завершения
    \item \texttt{Makefile} для удобства сборки и запуска
\end{itemize}

Для обмена данными между клиентом и сервером используется следующая структура:

\begin{lstlisting}[language=C, caption=Структура данных для обмена]
typedef struct {
    int number;         // Сгенерированное случайное число
    int ready;          // Флаг готовности (1 - число готово, 0 - нет)
    int terminate;      // Флаг завершения (1 - завершить работу, 0 - продолжать)
    pid_t client_pid;   // PID клиента для варианта с сигналами
} SharedData;
\end{lstlisting}

Общий алгоритм работы системы:
\begin{enumerate}
    \item Сервер и клиент запускаются независимо.
    \item Оба процесса подключаются к разделяемой памяти и инициализируют семафоры.
    \item Клиент генерирует случайное число, записывает его в разделяемую память и устанавливает флаг \texttt{ready}.
    \item Сервер считывает число из разделяемой памяти, выводит его и сбрасывает флаг \texttt{ready}.
    \item Процесс повторяется до тех пор, пока не будет получен сигнал завершения.
    \item При завершении работы одного из процессов освобождаются все ресурсы.
\end{enumerate}

\subsection{Варианты корректного завершения работы}

\subsubsection{Вариант 1: Завершение с использованием сигналов}

В этом варианте для завершения используются сигналы операционной системы. Клиент записывает свой PID в разделяемую память. Когда сервер решает завершить работу, он отправляет клиенту сигнал \texttt{SIGTERM}.

Преимущества:
\begin{itemize}
    \item Простота реализации — использование стандартных механизмов ОС.
    \item Прямое уведомление процесса о необходимости завершения.
\end{itemize}

Недостатки:
\begin{itemize}
    \item Требуется знать PID процесса для отправки сигнала.
    \item При неожиданном завершении одного процесса другой может не получить сигнал.
\end{itemize}

\begin{lstlisting}[language=C, caption=Фрагмент кода сервера (отправка сигнала клиенту)]
// Отправляем сигнал клиенту для завершения
if (shared_data->client_pid > 0) {
    printf("Сервер: отправка сигнала завершения клиенту (PID: %d)\n", 
           shared_data->client_pid);
    kill(shared_data->client_pid, SIGTERM);
}
\end{lstlisting}

\begin{lstlisting}[language=C, caption=Обработчик сигнала в клиенте]
void signal_handler(int sig) {
    done = 1;
}

// ...

// Установка обработчика сигнала
struct sigaction sa;
memset(&sa, 0, sizeof(sa));
sa.sa_handler = signal_handler;
sigaction(SIGTERM, &sa, NULL);
\end{lstlisting}

\subsubsection{Вариант 2: Завершение с использованием флага в разделяемой памяти}

В этом варианте в разделяемой памяти предусмотрен флаг \texttt{terminate}, который устанавливается процессом, который решает завершить работу. Другой процесс периодически проверяет этот флаг и также завершает работу при его установке.

Преимущества:
\begin{itemize}
    \item Не требуется знать PID процесса для уведомления о завершении.
    \item Работает даже если процессы запускаются не одновременно.
\end{itemize}

Недостатки:
\begin{itemize}
    \item Требуется явная проверка флага завершения в цикле обработки.
    \item При неожиданном завершении одного процесса (например, при сбое) флаг может не быть установлен.
\end{itemize}

\begin{lstlisting}[language=C, caption=Проверка флага завершения]
// Проверка флага завершения
if (shared_data->terminate || done) {
    printf("Клиент: получен сигнал завершения\n");
    break;
}
\end{lstlisting}

\begin{lstlisting}[language=C, caption=Установка флага завершения]
// Установка флага завершения для другого процесса
sem_wait(sem_producer);
shared_data->terminate = 1;
sem_post(sem_consumer);
\end{lstlisting}

\subsubsection{Вариант 3: Завершение с использованием дополнительного семафора}

В этом варианте используется дополнительный семафор \texttt{sem\_termination}. Когда один из процессов решает завершить работу, он увеличивает значение этого семафора. Другой процесс периодически проверяет значение семафора и завершает работу, когда оно становится больше нуля.

Преимущества:
\begin{itemize}
    \item Семафор сохраняется в системе даже после завершения процесса.
    \item Не требуется постоянного доступа к разделяемой памяти для проверки флага.
    \item Более надежный механизм при неожиданном завершении одного из процессов.
\end{itemize}

Недостатки:
\begin{itemize}
    \item Требуется дополнительный семафор.
    \item Необходимость его явной проверки.
\end{itemize}

\begin{lstlisting}[language=C, caption=Проверка семафора завершения]
// Проверка семафора завершения без блокировки
int termination_value;
sem_getvalue(sem_termination, &termination_value);
if (termination_value > 0) {
    printf("Клиент: получен сигнал завершения через семафор\n");
    break;
}
\end{lstlisting}

\begin{lstlisting}[language=C, caption=Сигнализация о завершении через семафор]
// Сигнализируем о завершении через семафор
sem_post(sem_termination);
\end{lstlisting}

\subsection{Сравнение вариантов завершения}

\begin{table}[h]
\centering
\begin{tabular}{|l|l|l|}
\hline
\textbf{Вариант} & \textbf{Преимущества} & \textbf{Недостатки} \\
\hline
Сигналы & 
\begin{tabular}[c]{@{}l@{}}• Стандартный механизм ОС\\• Прямое уведомление\end{tabular} & 
\begin{tabular}[c]{@{}l@{}}• Необходимость знать PID\\• Возможна потеря сигнала\end{tabular} \\
\hline
Флаг в памяти & 
\begin{tabular}[c]{@{}l@{}}• Не требует PID\\• Универсальность\end{tabular} & 
\begin{tabular}[c]{@{}l@{}}• Необходимость проверки\\• Проблемы при сбоях\end{tabular} \\
\hline
Семафор & 
\begin{tabular}[c]{@{}l@{}}• Надежность\\• Сохраняется в системе\end{tabular} & 
\begin{tabular}[c]{@{}l@{}}• Дополнительный ресурс\\• Явная проверка\end{tabular} \\
\hline
\end{tabular}
\caption{Сравнение вариантов завершения работы}
\label{tab:comparison}
\end{table}

\section{Результаты работы}

При запуске программы с использованием любого из трех вариантов мы получаем следующее поведение:

\begin{lstlisting}[language=bash, caption=Пример вывода при выполнении (вариант с сигналами)]
# Запуск сервера
$ ./server_signal
Сервер запущен. Нажмите Ctrl+C для завершения.

# Запуск клиента (в другом терминале)
$ ./client_signal
Клиент запущен (PID: 12345). Нажмите Ctrl+C для завершения.
Клиент: сгенерировано число 42
Сервер: получено число 42
Клиент: сгенерировано число 17
Сервер: получено число 17
...

# При нажатии Ctrl+C на сервере
Сервер: завершение работы...
Сервер: отправка сигнала завершения клиенту (PID: 12345)
Сервер: ресурсы освобождены

# В терминале клиента
Клиент: завершение работы...
\end{lstlisting}

Аналогичное поведение наблюдается и при запуске с другими вариантами завершения, с небольшими отличиями в сообщениях.

\section{Выводы}

В ходе выполнения лабораторной работы были успешно разработаны программы клиента и сервера, взаимодействующие через разделяемую память POSIX. Основные результаты работы:

\begin{enumerate}
    \item Реализован механизм обмена данными между процессами с использованием разделяемой памяти.
    \item Обеспечена синхронизация доступа к разделяемой памяти с помощью семафоров.
    \item Разработаны три различных варианта корректного завершения работы:
    \begin{itemize}
        \item Завершение с использованием сигналов
        \item Завершение с использованием флага в разделяемой памяти
        \item Завершение с использованием дополнительного семафора
    \end{itemize}
    \item Проведено сравнение реализованных вариантов, выявлены их преимущества и недостатки.
\end{enumerate}

Каждый из реализованных вариантов имеет свои особенности и может быть применен в зависимости от конкретных требований к системе. Вариант с сигналами является наиболее простым, но требует знания PID процесса. Вариант с флагом в разделяемой памяти более универсален, но требует постоянной проверки. Вариант с семафором является наиболее надежным, особенно в случае неожиданного завершения одного из процессов.

В реальных системах часто используется комбинация этих подходов для обеспечения максимальной надежности и гибкости.

\section{Приложение: Исходные коды программ}

Полный исходный код программ, включая заголовочные файлы, реализации клиентов и серверов для всех трех вариантов, а также Makefile, доступен в отдельных файлах.

\end{document}
